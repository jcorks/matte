\documentclass[12pt,letterpaper]{report}
\usepackage{hyperref}
\hypersetup{
    colorlinks,
    citecolor=black,
    filecolor=black,
    linkcolor=black,
    urlcolor=black
}
\author{Johnathan Corkery}
\title{Matte Language Specification\\
\small Version 1.0 Draft
}

\begin{document}
\maketitle
\tableofcontents


\chapter{Introduction}
\section{Background}    
\section{Intent}

The development of Matte as a language has a number of priorities which are 
considered as the deveopment process of the language continues:

- Usefulness. The language prioritizes having as few features as possible 
  while keeping it still useful. Features are added out of necessity for a reasonable, 
  object-oriented environment rather than an over-prediction of need.

- Simplicity. Matte focuses on keeping vagueness and esotericism to a minimum. One 
  of the reasons that Matte was developed was out of frustration of other 
  languages having a glut of features rather than being a focused tool. The 
  simplicity of the language is, in hope, extended to the ease-of-use of the language.

- Portability. While this document does not necessarily focus on implementation 
  of the language, portability is a primary concern of the language. 
  The intent of the language is to use it in a program-embedded setting, 
  where Matte is not the only language being used to develop a piece of software. 
  While Matte can, hypothetically, be used for any programming task, its intent is 
  for Matte to be relatively simple to implement. 


\section{Document Scope}
\section{Terminology}
\section{Concepts}
\chapter{Variables}
\section{Mutable}
\section{Immutable}
\section{Garbage Collection}
\section{Types}
\subsection{Empty}
\subsection{Boolean}
\subsection{Number}
\subsection{String}
\subsection{Object}
\subsubsection{Storage Semantics}
\subsubsection{Access Methods}
\subsubsection{Keys and Values}    
\subsubsection{Generic Operator Overloading}    
\subsubsection{Access Overloading}
\subsubsection{Implicit Type Conversion}
\subsubsection{Preservation}
\subsection{Type}            
\subsubsection{Explicit Conversion}
\subsubsection{Creating Types}
\subsubsection{Instantiation}
\subsection{Implicit Conversion}
\chapter{Expressions}
\section{Operators}
\subsection{Precedence}
\section{Literals}
\subsection{Strings}
\subsection{Object}
\subsection{List Objects}
\chapter{Functions}
\section{Statements}
\section{Lexical Scoping}
\section{Self-Refrentiability}
\section{The Source Function}
\section{As Expressions}
\section{Dash Syntax}
\chapter{Flow Control}
\section{when}
\section{match}
\section{loop}
\section{for}
\section{foreach}
\section{if}
\chapter{Introspection}
\chapter{External Interaction}
\section{import}
\section{getExternalFunction}
\chapter{Built-in Modules}
\section{Class}
\subsection{Introduction}
\subsection{Instantiation}
\subsection{Interface}
\subsection{Inhertiance}
\subsection{Automatic Pooling}
\section{Array}
\section{String}
\section{Enum}
\section{JSON}


\end{document}