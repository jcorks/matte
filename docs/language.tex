\documentclass[12pt,letterpaper]{report}
\usepackage{hyperref}
\hypersetup{
    colorlinks,
    citecolor=black,
    filecolor=black,
    linkcolor=black,
    urlcolor=black
}
\author{Johnathan Corkery}
\title{Matte Language Specification\\
\small Version 1.0 Draft
}

\begin{document}
\maketitle
\tableofcontents


\chapter{Introduction}
\section{Background}  

With the proliferation of software and availability of hardware to run it, a
plethora of tools have been continually made and refined to meet the 
continual challenges of software development. 

The popularity and widespread use of languages such as Javascript and 
Python suggest that every-day software challenges can be approached using 
software solutions that need not be tied to lower-level hardware while 
still providing the necessary performance to accomplish tasks.

However, many of these languages attempt to be "jack-of-all-trades", often 
inflating the languages to the point that every day use for users will involve
independent sets of language features, often using features only known to a few.

Rather than depending on a language to provide every tool for every situation,
the language, Matte, was intended to have straightforward, regular, and 
relatively simple operation, allowing for better colaboration, easier 
extension, and simple implementation and maintenance.
  
\section{Intent}

The development of Matte as a language has a number of priorities which are 
considered as the deveopment process of the language continues:

- Usefulness. The language prioritizes having as few features as possible 
  while keeping it still useful. Features are added out of necessity for a reasonable, 
  object-oriented environment rather than an over-prediction of need.

- Simplicity. Matte focuses on keeping vagueness and esotericism to a minimum. One 
  of the reasons that Matte was developed was out of frustration of other 
  languages having a glut of features rather than being a focused tool. The 
  simplicity of the language is, in hope, extended to the ease-of-use of the language.

- Portability. While this document does not necessarily focus on implementation 
  of the language, portability is a primary concern of the language. 
  The intent of the language is to use it in a program-embedded setting, 
  where Matte is not the only language being used to develop a piece of software. 
  While Matte can, hypothetically, be used for any programming task, its intent is 
  for Matte to be relatively simple to implement. 

- Extensibility. The language itself is kept minimal to allow for extension of its 
  usefulness.

\section{Document Scope}

This document intends to:
- give specifics on the use and behavior of the Matte language. 
- inform those who wish to implement the language.
- define standards for the language's behavior.

This document does not:
- define the implementation details of the language, i.e. the 
  structure of a VM, definition of an instruction set, or bytecode blob layouts.
- give domain-specific information or behavior 


\section{Conventions}

- All terms that refer to variables / concepts / etc that exist within the language verbatim will 
  be displayed in italics. For example, the keyword \textit{foreach} shall be displayed as such 
  throughout the document.
  
- Each syntactic concept shall have included with it its formal grammars. Within this specification 
  Wirth Syntax Notation (WSN) will be used for simplicity.


\section{Terminology}

- Compiler: The environment in which Matte source is interpreted. This may be a program in which Matte is running, part of a virtual machine, etc. The runtime is the software and/or hardware that must conform to this document to interpret Matte source correctly and consistently.

- Empty: special value that represents no value. 

- Runtime: The environment in which Matte is running. This may be a program in which Matte is running, a virtual machine, etc. The runtime is the software and/or hardware that must conform to this document to execute Matte source correctly and consistently.

- Variable: Points to a value.

- Value: a datum of a certain type.


\section{Concepts}
\chapter{Variables}

Variables act as conceptual storage points for working data values. Variable must be declared before usage.
A variable may live past its original scope lifetime (\autoref{sec:Lexical Scoping}) by being used, captured (\autoref{sec:Reference Capture}), and stored within Objects.

A variable may either be mutable or immutable. Variable values have types of their own, but 
types are determined by the values and expressions that produce them rather than by the 
variable.

\section{Mutable}

Within any function, mutable variables may be declared with a declaration statement.

    <mutable-declaration> ::= "@" <variable-identifier> <statement-end> | "@" <variable-identifier> "=" <expression> <statement-end> 

    <variable-identifier> ::= <any-letter-or-underscore> <variable-identifier>
        

Mutable variables can point to different values over its lifetime by reassignment.

\section{Immutable}
\section{Garbage Collection}
\section{Types}
\subsection{Empty}
\subsection{Boolean}
\subsection{Number}
\subsection{String}
\subsection{Object}
\subsubsection{Storage Semantics}
\subsubsection{Access Methods}
\subsubsection{Keys and Values}    
\subsubsection{Generic Operator Overloading}    
\subsubsection{Access Overloading}
\subsubsection{Implicit Type Conversion}
\subsubsection{Preservation}
\subsection{Type}            
\subsubsection{Explicit Conversion}
\subsubsection{Creating Types}
\subsubsection{Instantiation}
\subsection{Implicit Conversion}
\chapter{Expressions}
\section{Operators}
\subsection{Precedence}
\section{Literals}
\subsection{Strings}
\subsection{Object}
\subsection{List Objects}
\chapter{Functions}
\section{Statements}
\section{Lexical Scoping}
\section{Reference Capture}
\section{Self-Refrentiability}
\section{The Source Function}
\section{As Expressions}
\section{Dash Syntax}
\section{Type-Strict Functions}
\chapter{Flow Control}
\section{when}
\section{match}
\section{loop}
\section{for}
\section{foreach}
\section{if}
\section{throw/catch}

The "throw and catch" paradigm can serve as an alternative way to exchange information within 
a Matte program. \textit{throw()} is a built in function that, when called, signals to the runtime
that a value needs to be caught. Once the runtime is aware of the catchable value, all current 
functions running on the callstack will terminate from top to bottom (a.k.a "unwind") until a 
stackframe function is detected to have a \textit{catch} property that is callable.

When such a \textit{catch} property is found, the function is called with the first argument of the \textit{throw}
call passed as the only argument. The return value of the \textit{.catch} property function will be 
the new return value of the stackframe function.



\subsection{error}

Often, language use the "throw and catch" paradigm for error handling. Such can be, and is, done 
for the Matte language. The \textit{error()} function exists for this purpose. It functions 
equivalently to \textit{throw()}, except when a \textit{catch} is found, the first argument 
to the \textit{.catch} is a runtime-specific object containing data about where the error 
was first created. 

The format of this object is not formalized by the Matte language, except for the 
\textit{.data} property, which will always be the contents of the first argument 
of the \textit{error()} call. Runtime implementations are 

\chapter{Introspection}
\chapter{External Interaction}
\section{import}
\section{getExternalFunction}
\chapter{Built-in Modules}
\section{Class}
\subsection{Introduction}
\subsection{Instantiation}
\subsection{Interface}
\subsection{Inhertiance}
\subsection{Automatic Pooling}
\section{Array}
\section{String}
\section{Enum}
\section{JSON}


\end{document}
